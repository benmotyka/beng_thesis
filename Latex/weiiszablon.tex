%%%%%%%%%%%%%%%%%%%%%%%%%%%%%%%%%%%%%%%%%%%%%%%%%%%%%%%%%%%%%%%%%
%%% %
%%% % weiiszablon.tex
%%% % The Faculty of Electrical and Computer Engineering
%%% % Rzeszow University Of Technology diploma thesis Template
%%% % Szablon pracy dyplomowej Wydziału Elektrotechniki 
%%% % i Informatyki PRz
%%% % June, 2015
%%%%%%%%%%%%%%%%%%%%%%%%%%%%%%%%%%%%%%%%%%%%%%%%%%%%%%%%%%%%%%%%%

\documentclass[12pt,twoside]{article}

\usepackage{weiiszablon}

\author{Motyka Beniamin}

% np. EF-123456, EN-654321, ...
\studentID{EF-160780}

\title{Bezpieczeństwo IT dla firm - opis i implementacja}
\titleEN{Temat pracy po angielsku}


%%% wybierz rodzaj pracy wpisując jeden z poniższych numerów: ...
% 1 = inżynierska	% BSc
% 2 = magisterska	% MSc
% 3 = doktorska		% PhD
%%% na miejsce zera w linijce poniżej
\newcommand{\rodzajPracyNo}{1}


%%% promotor
\supervisor{dr Michał Piętal}
%% przykład: dr hab. inż. Józef Nowak, prof. PRz

%%% promotor ze stopniami naukowymi po angielsku
\supervisorEN{(academic degree) Imię i nazwisko opiekuna}

\abstract{Treść streszczenia po polsku}
\abstractEN{Treść streszczenia po angielsku}

\begin{document}

% strona tytułowa
\maketitle

\blankpage

% spis treści
\tableofcontents

\clearpage
\blankpage

\clearpage
\section{Wprowadzenie}

W dzisiejszych czasach śmiało można stwierdzić, iż	Internet stał się ważną częścią ludzkiego istnienia. Niewątpliwy wpływ na ten stan rzeczy miała pandemia COVID-19 - sprawiła ona bowiem, że pewne dziedziny życia, takie jak dydaktyka czy praca wykonywana umysłowo, przeszły swoistą transformację. Miejsca, w których spotykali się studenci wraz z wykładowcami, czy pracownicy w biurze, stały się puste. Zastąpiła je komunikacja zdalna -- przez Internet.

Fakt, iż ludzkość została zmuszona, by przenieść znaczną część swojego funkcjonowania w sieć, niesie ze sobą poważne konsekwencje. Szybkie -- jak do tej pory -- tempo rozwijania się technologii informatycznych stało się nieporównywalnie bardziej dynamiczne, a co się z tym wiąże, obecne w sieci liczne zagrożenia stały się coraz powszechniejsze i trudniejsze w identyfikacji.

[coś tu jeszcze będzie]

\subsection{Cel i zakres pracy}

Celem niniejszej pracy inżynierskiej jest wyeksponowanie, opis oraz implementacja najbardziej pospolitych zagrożeń i luk bezpieczeństwa w Internecie nie tylko dla zwykłych użytkowników, ale również dla małych i średnich przedsiębiorstw. Dzięki temu, że powyższa idea zostanie zrealizowana w formie aplikacji Internetowej, istnieje realna szansa na zwiększenie świadomości społecznej, edukację oraz poprawę zabezpieczeń systemów teleinformatycznych i infrastruktury sieciowej. Zakresem pracy są takie zagadnienia jak:
\begin{itemize}
\item Przegląd i dokumentacja zagrożeń i luk bezpieczeństwa.
\item Implementacja aplikacji e-learningowej przy użyciu technologii opisanych w kolejnym rozdziale.
\item Zasugerowanie potencjalnych rozwiązań na opisane cyberzagrożenia.
\item Przedstawienie wniosków i implikacji płynące z powyższych.
\end{itemize} 
[coś tu jeszcze będzie]



\clearpage

\section{Technologie wykorzystane w aplikacji}

\subsection{Node.js}
\subsubsection{Express}
\subsection{GraphQL}
\subsection{MongoDB}
\subsection{React.js}

\clearpage
\section{Zagrożenia bezpieczeństwa}

\subsection{Phishing - opis}
\subsection{Phishing - przykład implementacji}

\subsection{Ransomware - opis}
\subsection{Ransomware - przykład implementacji}

\subsection{Keylogger - opis}
\subsection{Keylogger - przykład implementacji}

\subsection{Wstrzyknięcie SQL - opis}
\subsection{Wstrzyknięcie SQL - przykład implementacji}

\subsection{DDOS - opis}
\subsection{DDOS - przykład implementacji}

\subsection{Ataki XSS - opis}
\subsection{Ataki XSS - przykład implementacji}

\subsection{Zdalne wykonywanie kodu - opis}
\subsection{Zdalne wykonywanie kodu - przykład implementacji}

\subsection{Spoofing - opis}
\subsection{Spoofing - przykład implementacji}

\clearpage
\section{Podsumowanie i wnioski końcowe}



\clearpage

\addcontentsline{toc}{section}{Literatura}

\begin{thebibliography}{4}
\bibitem{str} http://weii.portal.prz.edu.pl/pl/materialy-do-pobrania. Dostęp 5.01.2015.
\bibitem{Jakubczyk1997} Jakubczyk T., Klette A.: Pomiary w akustyce. WNT, Warszawa 1997.
\bibitem{Barski2011} Barski S.: Modele transmitancji. Elektronika praktyczna, nr 7/2011, str. 15-18.
\bibitem{dokum} Czujnik S200. Dokumentacja techniczno-ruchowa. Lumel, Zielona Góra, 2001.
\bibitem{Pawluk2001} Pawluk K.: Jak pisać teksty techniczne poprawnie, Wiadomości Elektrotechniczne, Nr 12, 2001, str. 513-515.
\end{thebibliography}

\clearpage

\makesummary

\end{document} 
